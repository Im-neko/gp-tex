\section{はじめに}
ドローンをRGBカメラで制御しようという考えに至った理由として,映像処理技術の向上,ドローン製作部品が現時点では比較的高価格であること,があげられる.高度維持やGPSによる位置補正など一般的な機能を備えた安定飛行できるドローンを一機製作しようとすれば,センサーの類だけで数万円はかかってしまう.といった事が挙げられる.

そこで,既に小型化,低価格化が進んでいるRGBカメラを用いた画像処理による飛行経路探索を行う.

加えて,LIDARなどのセンサーで得られるデータは点群データであり,周囲の物体の表面の情報などは得られない上に現在の主な点群データを用いた自立飛行手法では点群データを基にコンピュータ内部で3次元マップを作成する必要があり,ハイパワーなマシンが求められる.

一方でRGBカメラによる映像や少ない点群データを基にした自立飛行手法も増えつつある\cite{Nanomap}\cite{SfMDrone}\cite{DeepDrone}.

\subsection{研究目的}
本研究ではRGBカメラを用いたレースドローンの自立飛行を最終的な目的としており,TERMではゲートに見立てた黄色のフラフープを通過できるシステムの構築までを目指す.

また,本システムは映像データの取得とコンピュータによる操作さえ可能であれば機種を問わず利用可能なシステムを目指す.

将来的には本手法単体で飛行経路選択をする以外にも,他のLIDARなどのセンサーも合わせて使用するような制御系における,制御の一助となるような使用を想定している.
