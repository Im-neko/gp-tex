\section{はじめに}
% ドローンをRGBカメラで制御しようという考えに至った理由として,映像処理技術の向上,ドローン製作部品が現時点では比較的高価格であること,があげられる.高度維持やGPSによる位置補正など一般的な機能を備えた安定飛行できるドローンを一機製作しようとすれば,センサーの類だけで数万円はかかってしまう.
% そこで,既に小型化,低価格化が進んでいるRGBカメラを用いた画像処理による飛行経路探索を行う.
% 加えて,LIDARなどのセンサーで得られる情報で障害物等自体の意味解釈まで行う事は困難である.そこで,RGBカメラを用いて周囲の映像を取得し周辺にどのような物体が存在しているのかを検出しながらの飛行が求められる.
初めてドローンを操縦する人や,まだあまり操縦になれない人がドローンの操縦を行う際操縦者はドローンの操縦のみで手いっぱいになってしまう光景が多々見受けられる.しかし,実際に操縦する際には周囲の人の動きや到達目標点など操縦以外にも気にかける必要のある事由は多い.
この時,屋外など広い場所であれば人の立ち入りを制限したり,安全飛行支援システムの整っている高機能なドローンを使用すればこれらの問題は解決し易いが,これらの準備は非常に手間がかかり,飛行するための準備コストも高い.そこで,屋内等で手軽に飛ばせるトイドローンでも現在よりも飛行支援を充実させたいと考えている.現在はイメージセンサと超音波センサなどによる安定飛行支援のみで,障害物検知を搭載しているトイドローンは少ない.そのため,障害物に対する飛行支援システムの構築を目指す.

% \subsection{背景}
% 2010年頃から現在にかけてParrot社による家庭用ドローンの発売などもありドローンの認知と研究開発は加速している.
% これに付随して,自律飛行に関しても研究が盛んに行われているが,現在行われている自律飛行ではLIDARなど比較的高価なものを使用している例が多い.
% また広く流通しているこれらのセンサーは屋内で多く使用される機体に対して大きいものが多く,小型のドローンでの自律飛行には用い難い.一方で,この問題は小型化,高機能化が日々行われているセンサー類においては大きな問題とはなり得ない.
% しかし,今後複数のドローンが同じ空間で飛び交う状況や,人やその他動的な物体がある空間での飛行が想定される中では周囲の映像に映る物体の意味解釈をした上での飛行モデルの作成は非常に重要となると考え本研究に取り組むに至った.
%

\subsection{研究目的}
% 本研究の目的はRGBカメラによるドローンの自律飛行,その中でも自機前方カメラに映る物体の中に動的な物体があった場合にそのその飛行経路を物体に合わせて変更する.という所を最終到達点としている.
%
% また,将来的には本手法単体で飛行経路選択をする以外にも,他のLIDARなどのセンサーも合わせて使用するような制御系における,制御の一助となるような使用を想定している.
障害物回避に対する飛行支援システムが搭載されていないが映像のリアルタイム伝送が可能なカメラを搭載しているドローンにおいてその映像を元に操縦者が操縦に集中しつつもドローン前方に注意すべき対象が存在する場合にその表示を行えるシステムの構築を目指す.

また,特定の機体に依存するシステムではなく,トイドローンにおいてはスマートフォンに映像を伝送するものが多く,スマートフォンに映像伝送を行う種類のドローンであればこのシステムを利用できるように構築する.
