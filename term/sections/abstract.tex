\begin{abstract}
% 2018年現在自律飛行で多く使用されるLIDARなどのセンサーは他の超音波やRGBカメラに比べて高価であり,またサイズも広く流通しているものは大きい物が多い.
2018年現在ドローンの個人所有も珍しい事柄ではなくなり,より多くの人が操縦をする機会が増えてきた.
しかし,その操作は複雑であり,操縦に集中しながら周囲に気を配る事は初心者にとって難易度の高い物となる.
そこで現在は多くの飛行システムに超音波センサーやイメージセンサなどを用いた安定飛行や障害物への衝突回避機能が搭載されているが,小型のといドローンにも搭載されている事は少なく,加えて搭載されていても物体自体の属性を考慮する事は難しい.同じ距離の障害物であっても形や物の硬さ,移動物体なのか設置物なのか,衝突時の被害の大きさなどが異なる.そこで今回はドローン積載カメラによる映像を用いた周囲の障害物の検知による飛行支援システムを構築する.
% また,今回は研究を進める上での状況設定として考慮事項が少なく,再現性のある環境を作れるようにする為にドローンレースを前提としている.
% 流れとしては実際にドローンレースで用いられるようなゲートを配置しそれらの理想形となる通過軌道を正としてシミュレーションを中心に学習を行う.
% また,この時複数のコースを同時に学習させる事で汎用性を持たせられるという先行研究もあり本研究でも同様に進めるものとする.

\end{abstract}
