\begin{abstract}
2010年代以降ドローン(UAV)の産業,個人利用が増えて来ている.
産業では空撮,測量,物資輸送等幅広い活用に向けての研究開発が行われており,個人でも空撮やレース目的でのドローン個人所有が進んでいる.

その中で自立飛行に関する研究は非常に注目を集めており,様々な活用法の中で自立飛行手法が提案されている.
2018年現在での自律飛行モデルで多く使用される手法では経路探索に使うデータにLIDARなどの比較的高価なセンサーを用いたデータを利用する物が多い.
しかしLIDARだけでは周囲にある物体を点群データとしてしか捉える事ができず,現実世界で飛行させるには情報不足である節がある.
加えて現在の主立った点群データを処理する手法ではドローンに搭載出来るようなオンボードマシンでは処理負荷が高く,ハイスペックな専用のボードが必要になる.

そこで,本研究では搭載コストが低く,物体の表面の情報も得られるRGBカメラを利用した自立飛行モデルを構築する.
また,状況設定として現在のドローン活用方法の中でも他のセンサーが少なく,ドローン本体からの映像だけで操縦するドローンレースを対象として,RGBカメラを用いたドローンレースの自動化を目指す.
TERMではゲートに見立てた黄色のフラフープを通過できるシステムの構築までを目指す.

\end{abstract}
