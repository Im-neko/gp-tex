\section{トラッカー部}
\label{implement_tracker}
トラッカー部の実装にはOpenCVを用いる.

QRコードの検出とデコードにはOpenCVに用意されているDetectQRクラスを利用し,そこから得られたQRコードの四角に対してPnPによる3次元再構成を行った.
PnPにはOpenCVに用意されているsolvePnPRansac関数を利用した.

これによりQRコードよりデコードされた次のQRコードの位置情報とQRコードに対するカメラの回転ベクトル,並進ベクトルが得られるのでこれらの情報をナビゲーション部\ref{implement_navigation}にて処理をする

実際にスマートフォン上に表示したQRコードに対して姿勢推定結果を描画したものを以下の図\ref{pnp_qr_img}に示す

\begin{figure}[htbp]
  \begin{center}
    \includegraphics[clip,width=15.0cm]{img/pnp_qr.png}
    \caption{QRコードの姿勢推定結果}
    \label{pnp_qr_img}
  \end{center}
\end{figure}

ソースコードの該当部分をそれぞれ以下のプログラム\ref{src_pnp}に示す.
\begin{lstlisting}[caption=pnp method,label=src_pnp]
def pnp_qr(self, frame):                                                    
    img = cv2.cvtColor(frame, cv2.COLOR_RGB2BGR)                            
                                                                            
    qr = cv2.QRCodeDetector()
    data, points, _ = qr.detectAndDecode(img)
    if data:                                                                
        self.qr_data = self._qr_validator(json.loads(data))
        _, origin_rvec, origin_tvec, inliers = cv2.solvePnPRansac(
            self.objp, points, self.mtx, self.dist)
        if not self.is_moving:
            self.target_pos = self.qr_data
            tvec = origin_tvec * self.UNIT_SIZE
            rvec = origin_rvec * self.RAD_UNIT
            self.drone_pos = {
                'x': tvec[0],                                               
                'y': tvec[1],                                               
                'z': tvec[2],                                               
                'r': rvec[1]                                                
            }                                                               
        imgpts, jac = cv2.projectPoints(
            self.axis, origin_rvec, origin_tvec, self.mtx, self.dist)
        img = self.draw(img, points, imgpts)
                                                                            
        cv2.imshow('img', img)
        cv2.waitKey(10)             
    else:                                                                   
        cv2.imshow('img', img)                                              
        cv2.waitKey(10)                                                     
    self.out.write(img) 
\end{lstlisting}
