\section{トラッカーノード}
\label{implement_tracker}
トラッカーノードの実装にはOpenCVを用いてQRコードの検出と3次元再構成を行う.

ソースコードの該当部分をそれぞれ以下のプログラム\ref{src_pnp}に示す.

\begin{lstlisting}[caption=pnp\_qr.py,label=src_pnp]
# TODO: rosパッケージとしてまとめる
import cv2
import numpy as np

def draw(img, corners, imgpts):
    corner = tuple(corners[0].ravel())
    img = cv2.line(img, corner, tuple(imgpts[0].ravel()), (255,0,0), 5)
    img = cv2.line(img, corner, tuple(imgpts[1].ravel()), (0,255,0), 5)
    img = cv2.line(img, corner, tuple(imgpts[2].ravel()), (0,0,255), 5)
    return img

criteria = (cv2.TERM_CRITERIA_EPS + cv2.TERM_CRITERIA_MAX_ITER, 30, 0.001)
objp = np.zeros((2*2, 1, 3), np.float32)
objp[:,:,:2] = np.mgrid[0:2,0:2].T.reshape(-1,1,2)

axis = np.float32([[3,0,0], [0,3,0], [0,0,-3]]).reshape(-1,3)

# camera pamrameters
mtx = np.load("../calib/mtx.npy")
dist = np.load("../calib/dist.npy")

capture = cv2.VideoCapture(0)

while True:
    _, img = capture.read()
    qr = cv2.QRCodeDetector()
    data, points, _ = qr.detectAndDecode(img)
    if data:
        print(data, points)
        _, rvecs, tvecs, inliers = cv2.solvePnPRansac(objp, points, mtx, dist)
        imgpts, jac = cv2.projectPoints(axis, rvecs, tvecs, mtx, dist)

        img = draw(img,points,imgpts)

        cv2.imshow('img',img)
        k = cv2.waitKey(5) & 0xff
        if k == 's':
            cv2.imwrite(fname[:6]+'.png', img)
    else:
        cv2.imshow('img',img)
        k = cv2.waitKey(1) & 0xff
        if k == 's':
            cv2.imwrite(fname[:6]+'.png', img)
\end{lstlisting}
