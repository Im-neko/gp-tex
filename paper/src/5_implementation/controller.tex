\section{コントローラ部}
\label{implement_controller}
制御の流れとしては以下の流れをとった.

\begin{enumerate}
    \item QRコードが画角中央且つ距離85cmの位置に来るように推定値を用いて制御する,
    \item QRコードが上下左右距離が誤差±10cmになっていれば次点へ移動する.なっていなければ1の処理を行う
    \item QRコードの中の情報を利用して次の地点まで誘導する.
    \item 以下1,2,3の繰り返し
\end{enumerate}

この際推定誤差によって大きく外れた値が推定される事があり,毎フレーム推定と制御を行うと外れ値を制御に利用してしまうこととなる.
その為,今回は20フレーム分バッファを作り,その平均値を推定値として扱うようにした.
実際に移動する際には以下の\_move関数~\ref{move_method}を用いており,
次のcheck\_standby関数~\ref{check_standby}がTrueを返すようになるまでQRコードに対して誤差±10cmになるように位置の補正を行う

\begin{lstlisting}[caption=move method code,label=move_method]
def _move(self, data):          
    self.is_moving = True
    # move x                   
    if data["x"] >= 10:            
        self.drone.move_right(data["x"])
        time.sleep(data["x"]/self.SPEED + self.COMMAND_WAIT_BUFFER)
        print('sleep x: ', data["x"]/self.SPEED +
            self.COMMAND_WAIT_BUFFER)
    elif data["x"] <= -10:
        self.drone.move_left(-data["x"])
        time.sleep(-data["x"]/self.SPEED + self.COMMAND_WAIT_BUFFER)
        print('sleep x: ', -data["x"]/self.SPEED +                                                                                                           
            self.COMMAND_WAIT_BUFFER)
    # move y
    if data["y"] >= 10:
        self.drone.move_down(data["y"])
        time.sleep(data["y"]/self.SPEED + self.COMMAND_WAIT_BUFFER)
        print('sleep y: ', data["y"]/self.SPEED +
            self.COMMAND_WAIT_BUFFER)
    elif data["y"] <= -10:
        self.drone.move_up(-data["y"])
        time.sleep(-data["y"]/self.SPEED + self.COMMAND_WAIT_BUFFER)
        print('sleep y: ', -data["y"]/self.SPEED +
            self.COMMAND_WAIT_BUFFER)
    # move z
    if data["z"] >= 10:
        self.drone.move_forward(data["z"])
        time.sleep(data["z"]/self.SPEED + self.COMMAND_WAIT_BUFFER)
        print('sleep z: ', data["z"]/self.SPEED +
            self.COMMAND_WAIT_BUFFER)
    elif data["z"] <= -10:
        self.drone.move_backward(-data["z"])
        time.sleep(-data["z"]/self.SPEED + self.COMMAND_WAIT_BUFFER)
        print('sleep z: ', -data["z"]/self.SPEED +
            self.COMMAND_WAIT_BUFFER)
    # rotate r
    if data["r"] > 0:
        self.drone.rotate_cw(data["r"])
    else:
        self.drone.rotate_ccw(-data["r"])
    time.sleep(self.COMMAND_WAIT_BUFFER)
    # initialize values
    self.is_moving = False
    self.is_standby = False
    self.drone_pos_list = []
    self.drone_rotate_list = []
    return
\end{lstlisting}

\begin{lstlisting}[caption=check standby method, label=check_standby]
def check_standby(self):
    x_val = int(self.drone_pos["x"] - (self.QR_SIZE/2))
    y_val = int(self.drone_pos["y"] - (self.QR_SIZE/2))
    z_val = int(self.drone_pos["z"] - self.Z_SPACE)
    rotate_val = int(self.drone_pos["r"])
    self.is_standby = -self.permit_noise < x_val and self.permit_noise > x_val and \
        -self.permit_noise < y_val and self.permit_noise > y_val and \
        -self.permit_noise < z_val and self.permit_noise > z_val and \
        -self.permit_noise < rotate_val and self.permit_noise > rotate_val
    print('check: ', x_val, y_val, z_val, rotate_val)
    print('is_standby: ', self.is_standby)
\end{lstlisting}