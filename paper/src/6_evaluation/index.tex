\chapter{評価}
\label{evaluation}
本章では,提案システムの評価について述べる.

\section{評価方法}
倉庫棚を模した棚にQRコードを貼り付け\ref{course},実際に想定した経路を飛行できるかを確認した.
また,本手法の特徴として毎回QRコードの前で位置の補正を行う為他の手法よりも巡回に時間がかかる物と考えられる.
そこで,コースを巡回させ続け位置補正に要する時間を100回分計測した.
実際には以下のコースを用いて計測した.


\begin{figure}[htbp]
  \begin{center}
    \includegraphics[clip,width=15.0cm]{img/course.jpg}
    \caption{評価用コース}
    \label{course}
  \end{center}
\end{figure}

また,QRコードにはそれぞれ以下の情報が文字列で保存されている
\begin{enumerate}
    \item {x: 80}
    \item {x: 90}
    \item {x: 90}
    \item {y: 70}
    \item {x: -90}
    \item {x: -90}
    \item {x: -80}
    \item {y: -70}
\end{enumerate}

今回は使用機材の解像度の都合上QRコードに対して距離85cmを維持するようにした.

\section{評価結果}
今回倉庫棚を想定した棚での飛行実験は安定して想定通りの経路を飛行を行う事が出来た\footnote{\url{https://youtu.be/pw23Vq8DuHE}}.
具体的な飛行成績は以下の通りである.

\begin{table}[h]
    \caption{飛行結果}
    \label{table:fly_result}
    \centering
    \begin{tabular}{cccc}
        \hline
        試行回数 & 到達可否 & 周回時間[sec] & 平均位置補正時間[sec] \\
        \hline \hline
        1回目 & 可 & 214.58 & 12.22 \\
        2回目 & 可 & 199.19 & 10.08 \\
        3回目 & 可 & 225.43 & 12.35 \\
        4回目 & 可 & 242.86 & 14.20 \\
        5回目 & 可 & 289.67 & 19.44 \\
        6回目 & 可 & 294.40 & 19.97 \\
        7回目 & 可 & 285.03 & 18.93 \\
        8回目 & 可 & 265.28 & 16.76 \\

        9回目 & 可 & 214.58 & 12.22 \\
        10回目 & 可 & 214.58 & 12.22 \\
        \hline
    \end{tabular}
\end{table}

初回の位置合わせに時間がかかる事が多いものの,二つ目以降の位置合わせには時間差はあまり出なかった.
以下に90回分の位置補正にかかった時間の箱日げ図を示す.\ref{box_plot}

\begin{figure}[htbp]
    \begin{center}
      \includegraphics[clip,width=15.0cm]{img/timedata.png}
      \caption{位置補正に要した時間}
      \label{box_plot}
    \end{center}
  \end{figure}

%%% Local Variables:
%%% mode: japanese-latex
%%% TeX-master: "./thesis"
%%% End:
