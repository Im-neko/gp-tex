\chapter{評価}
\label{evaluation}
本章では,提案システムの評価について述べる.

\section{評価方法}
倉庫内を模した周回コースを作成し,QRコードの数を変化させた際の周回成功率と速度を計測する.
QRコードの数を増やせばコース上の任意のQRコードから次点のQRコードまでの距離も短くなり,
カメラの計測誤差等による経路誤差も小さくなり周回成功率も向上すると考えられる.
一方で本システムではQRコードを認識するとそこで一時停止する為,
QRコードの数を減らせば次点へ向かうまでの時間が短くなり周回速度が向上すると考えられる.



%%% Local Variables:
%%% mode: japanese-latex
%%% TeX-master: "./thesis"
%%% End:
