\chapter{結論}
\label{conclusion}

本章では,本研究のまとめと今後の課題を示す.

\section{本研究のまとめ}
% ここもうちょっと文章まともにする

今回のこのシステム実際に使う上では間違って他のQRコードを認識してしまったり,悪意ある人間によってQRコードがすり替えられていないかの検証システムが必要になってくる.
その場合にはjwtなどの方式でQRコード内のデータに署名をする.という事で対応できるが,悪意ある人間がその環境に存在した場合QRコードの場所が置き換えられる可能性や複製されてしまうと言う可能性は残る.

また,QRコードをLEDパネルで表示する方法\cite{led_qr}も考案されている為この方法を用いてQRコードを容易に変更・配置の記録をする事も可能になる.
一方でどこにどのQRコードが配置されているかはシステム全体として保持しておきたい情報でもあり,LEDパネルでQRコードを表示する手法は本研究との親和性が非常に高いと考えられる.

% 制御システムとしては環境地図作成や経路情報を保持しなくて良いと言った利点が残る上でQRコード自体の管理はソフトウェアで行うことができる.
% 本研究におけるQRコードに経路情報を載せる目的は倉庫での物品管理向けのナビゲーションシステムで経路情報を機体毎に管理するのは冗長であるという理由であった.


\section{本研究の課題}

%%% Local Variables:
%%% mode: japanese-latex
%%% TeX-master: "../thesis"
%%% End:
