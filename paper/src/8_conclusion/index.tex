\chapter{結論}
\label{conclusion}

本章では,本研究のまとめと今後の課題を示す.

\section{本研究のまとめ}
本研究で行った評価はあくまで今回利用した機材における性能評価に過ぎず,
実際にシステムとして実運用が可能かどうかは改めて実運用時の機材や環境での評価が必要であると考えられる.
今回利用した機材ではその仕様上20cm以下の細かい制御が出来ず,
位置補正にかかる時間はこの制御分解能にも大きく依存すると考えられるからである.
一方で,今回利用した機材は非常に簡易的な機材であるにも関わらず,
今回倉庫を想定して用意された棚では実用可能な精度で飛行する事が確認できた.
これらより本システムは倉庫の巡回タスクにおいては有効な手段である事を示した.
但し,利用する機材によって巡回速度や安定度は大きく左右する為,他の選択肢との比較を充分に行った上で採用する事が望ましい.


\section{本研究の課題}
\subsection{攻撃耐性}
本システム実際に使う上では間違って他のQRコードを認識してしまったり,
悪意ある人間によってQRコードがすり替えられていないかの検証システムが必要になってくる.
その場合にはjwtなどの方式でQRコード内のデータに署名をする.という事で対応できるが,
悪意ある人間がその環境に存在した場合QRコードの場所が置き換えられる可能性や複製されてしまうと言う可能性は残る.

\subsection{拡張性}
QRコードをLEDパネルで表示する方法\cite{led_qr}も考案されている為この方法を用いてQRコードを容易に変更・配置の記録をする事も可能になる.
どこにどのような内容のQRコードが配置されているかはシステム全体として保持しておきたい情報でもあり,LEDパネルでQRコードを表示する手法は本研究との親和性が非常に高いと考えられる.
また,この方法であれば動的にQRコードを変化させる事が可能であり,張り替えられたり複製されてしまう事への対策にもなり得る.
一方で本システムは経路情報の管理をシステムの外に出す事で制御システム自体の簡略化を図った物である事を考慮すると再び経路情報の管理用システムが必要となる点で本来の目的に反してしまう.
しかし制御システムから環境地図作成と専用機材を用意せず倉庫の巡回タスクを実施できるという点にメリットがある環境であれば検討の余地が充分にあると考えられる.

\subsection{精度向上}
本研究における実装は非常に簡易的な物であり,精度向上の余地は大いに残されている.
改善点としては以下の物が挙げられる
\begin{itemize}
    \item 使用機材の変更
    \item 推定パラメータの変更
    \item 推定値から外れ値の除去
\end{itemize}

\subsubsection{使用機材の変更}
今回使用している機材はその仕様上制御できる移動量が20cm~500cmまでとなっている.
その結果位置補正の際に本来望ましい移動量を超えて移動する場面が多く見受けられた.
例えばx軸方向に15cm移動したい場合にも最低値である20cm移動をしなければないというような場面である.
加えてカメラの画質によっても大きく精度に変化が現れる.
今回利用したカメラの解像度ではQRコード正面からの直線距離でおよそ120cmまでが認識できる限界の距離であった.


\subsubsection{推定パラメータの変更}

\subsubsection{推定値から外れ値の除去}



%%% Local Variables:
%%% mode: japanese-latex
%%% TeX-master: "../thesis"
%%% End:
