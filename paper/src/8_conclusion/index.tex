\chapter{結論}
\label{conclusion}

本章では,本研究のまとめと今後の課題を示す.

\section{本研究のまとめ}
本研究で行った評価はあくまで今回利用した機材における性能評価に過ぎず,
実際にシステムとして実運用が可能かどうかは改めて実運用時の機材や環境での評価が必要であると考えられる.
今回利用した機材ではその仕様上20cm以下の細かい制御が出来ず,
位置補正にかかる時間はこの制御分解能にも大きく依存すると考えられるからである.
一方で,今回利用した機材は非常に簡易的な機材であるにも関わらず,
今回倉庫を想定して用意された棚では実用可能な精度で飛行する事が確認できた.
これらより本システムは倉庫の巡回タスクにおいては有効な手段である事を示した.
但し,利用する機材によって巡回速度や安定度は大きく左右する為,他の選択肢との比較を充分に行った上で採用する事が望ましい.


\section{本研究の課題}
今回のこのシステム実際に使う上では間違って他のQRコードを認識してしまったり,悪意ある人間によってQRコードがすり替えられていないかの検証システムが必要になってくる.
その場合にはjwtなどの方式でQRコード内のデータに署名をする.という事で対応できるが,悪意ある人間がその環境に存在した場合QRコードの場所が置き換えられる可能性や複製されてしまうと言う可能性は残る.

また,QRコードをLEDパネルで表示する方法\cite{led_qr}も考案されている為この方法を用いてQRコードを容易に変更・配置の記録をする事も可能になる.
一方でどこにどのQRコードが配置されているかはシステム全体として保持しておきたい情報でもあり,LEDパネルでQRコードを表示する手法は本研究との親和性が非常に高いと考えられる.

%%% Local Variables:
%%% mode: japanese-latex
%%% TeX-master: "../thesis"
%%% End:
