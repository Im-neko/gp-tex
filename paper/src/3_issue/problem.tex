\section{本研究における問題定義}
現状の自律飛行システムを倉庫内での物品管理に用いる場合の問題点を列挙し,整理する.

\section{自己位置推定の現状の問題}
現状ロボットの自己位置推定に用いられる手法は大きく分けて2つあり,それぞれ問題がある.
一つはGPSを用いた自己位置推定を行う際の利用可能環境の限定性や精度である.もう一つはLIDAR等を用いた~\ref{slam}章で説明されるようなSLAMを行う際のその利用可能環境である.

\subsection{GPSを用いた自己位置推定における問題}
現在世界座標系における自己位置推定を行う上で頻繁に用いられるが,その利用可能環境には制限がある.
それは屋内や遮蔽物の多い環境で衛星電波を受信出来ず利用できないという事である.
加えて屋外の見通しの良い環境であっても誤差が平均して1~2m程あり\footnote{同研究室の水野史暁氏と竹瀬浩行氏と他数名によって慶應義塾大学湘南藤沢キャンパスWEST地区にて計測}細かい制御には向かない.

一方で現在GPSの精度向上や補助をする為のシステムが複数開発されている.
その中の一つがA-GPS\cite{agps}と呼ばれ主に携帯電話やスマートフォンに搭載されているシステムである.
A-GPSはGPSの信号強度が弱い場合や,建造物などの影響によるマルチパスが発生している際に利用可能なネットワークを利用して測位者の概ねの位置を取得する方式である.これによりGPS測位時間の短縮や測位可能なエリアの拡大を実現している.

他にJAXAが開発したIMES(Indoor MEssaging System)\cite{imes}という屋内にGPSと同じ電波を利用した送信機を設置し細かい位置測位を行うという物もある.
しかしこれらの方法は専用の送信機や通信機構を用意する必要がある.


\subsection{LiDARを用いたSLAMにおける問題}
屋内で自己位置推定を行う方法としてSLAMが挙げられる.

\subsubsection{Visual SLAMにおける問題}
Visual SLAMはその特性上比較的多くの計算リソースを必要とする.
計算機の小型化が進んでいるものの,手のひらサイズ程度の小型のロボット上で利用するには少々パワー不足となることが多い.
また,この問題を避ける為計算サーバのようなものを用意する事を考えると複数機体利用する場合にネットワーク帯域や計算リソースが問題となり,解消できるような環境を用意する場合結局費用面で実用には向かない.

\subsubsection{LiDAR SLAMにおける問題}
LiDAR SLAMはVisual SLAMに比べて計算リソースは不要なものの,そのセンササイズがカメラよりも大きい物が多い.
これは小型のロボットに積載するには向かず,ドローンのような積載重量によって航行性能に大きく影響が出るようなロボットでは尚のこと利用が難しい.
役割が異なるので純粋に価格で比較することは難しいが,センサ1個あたりの費用もSLAMに必要なスペックのものという前提で選出するとカメラモジュールよりも高くなる.
