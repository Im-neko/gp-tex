\section{本研究における問題定義}
現状の自律飛行システムを倉庫内での物品管理に用いる場合の問題点を列挙し,整理する.

\subsection{自己位置推定の現状の問題}
現状ロボットの自己位置推定に用いられる手法は大きく分けて2つあり,それぞれ問題がある.
一つはGPSを用いた自己位置推定を行う際の利用可能環境の限定性や精度である.もう一つはLIDAR等を用いた~\ref{slam}章で説明されるようなSLAMを行う際のその利用可能環境である.

\subsubsection{GPSを用いた自己位置推定における問題}
現在世界座標系における自己位置推定を行う上で頻繁に用いられるが,その利用可能環境には大きな制限がある.
現在GPSの精度向上や補助をする為のシステムが開発されている.
これはA-GPS\cite{agps}と呼ばれ主に携帯電話やスマートフォンに搭載されているシステムである.
A-GPSはGPSの信号強度が弱い場合や,建造物などの影響によるマルチパスが発生している際に利用可能なネットワークを利用して測位者の概ねの位置を取得する方式である.これによりGPS測位時間の短縮や測位可能なエリアの拡大を実現している.
他にJAXAが開発したIMES(Indoor MEssaging System)\cite{imes}


\subsubsection{LIDARを用いたSLAMにおける問題}

