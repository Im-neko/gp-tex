\subsubsection{SLAMを用いた自己位置推定}
\label{slam_problem}
屋内で自己位置推定を行う方法としてSLAMが挙げられる.

\subsubsection{Visual SLAM}
Visual SLAMはその特性上比較的多くの計算リソースを必要とする.加えて対象の輪郭などが確認できるくらいの明るさや色彩差がないと利用することが出来ない.
映像が理想的な状態で得られたとしても手のひらサイズ程度の小型のロボット上で利用するには少々パワー不足となることが多い.
更に今回想定しているようなリアルタイムでの計算が必要な状況においては計算リソースが非常に重要になる.
また,この問題を避ける為計算サーバのようなものを用意する事を考えると複数機体利用する場合にネットワーク帯域や計算リソースが問題となり,解消できるような環境を用意する場合倉庫の巡回タスクにおいては費用面で実用には向かない.
加えてVisual SLAMによる環境地図作成と自己位置推定の精度には周辺の映像から抽出できる特徴量によって大きく左右される.
例えば単色の背景が続くような環境や,ガラスなどの光透過性の高い環境では正確に環境地図の作成が行えない事が想定される.

\subsubsection{LiDAR SLAM}
LiDAR SLAMはVisual SLAMに比べて計算リソースは不要なものの,そのセンササイズがカメラよりも大きい物が多い.
これは小型のロボットに積載するには向かず,ドローンのような積載重量によって航行性能に大きく影響が出るようなロボットでは尚のこと利用が難しい.
役割が異なるので純粋に価格で比較することは難しいが,センサ1個あたりの費用もSLAMに必要なスペックのものという前提で選出するとカメラモジュールよりも高くなる.
加えてLiDAR SLAMにおいてもガラスなどの光透過性が高い環境では正確に環境地図の作成が行えない事が想定される.