\subsection{提案手法}
今回上述した課題を解決する為にQRコードを用いた経路情報に対してステートレスなナビゲーションシステムを考案した.
この方法には以下の利点が存在する
\begin{itemize}
    \item RGBカメラが積載されていれば利用可能
    \item QRコードを用意するのは他の手法に比べ容易
    \item QRコードは非対称図形のためQRコードに対する相対的な位置推定が可能
\end{itemize}
上述した課題をどのように解決するかをそれぞれ以下にまとめる.

\subsubsection{経路情報の管理}
今回経路情報の管理をソフトウェアでは行わないという方法で課題となっていた部分を解決する.
QRコードを一定間隔で棚に貼り付け,ドローンに積載されているカメラで読み取り経路情報を取得するという形をとる事にした.
QRコードには次に向かうべきQRコードの座標データが相対座標系で記録されており,ドローンはこの情報を元に飛行計画を作成する事となる.

\subsubsection{倉庫内での自己位置推定}
本手法ではQRコードに対する自身の相対的な位置をPnP問題として計算して推定する.
倉庫の巡回タスクにおいては倉庫全体絡みた自己位置推定はせずともタスクを遂行することが可能である.
経路情報の項で述べたようにQRコードに次に向かうべき座標データがその時点で観測しているQRコードを原点とする座標系で記載されている.
その為,QRコードに対する相対的な自己位置推定が出来れば想定される飛行経路に沿った飛行が可能となる.
しかし,次のQRコードに移動する途中でQRコードを認識出来ない区間はドローン自体のオドメトリに頼った自己位置推定となり使用する機材に大きく依存した精度での制御となる.
今回使用する機材ではモーターの回転数,ドローン下部に搭載された単眼カメラ,赤外線センサによるオドメトリでの自己位置推定を行っている.

\subsubsection{専用機材の用意}
既存手法ではLiDAR SLAMではLiDARセンサとそれを積載可能なサイズのドローン,
得られた点群データを処理できるだけの計算リソースが必要であった.
またVisual SLAMではカメラ映像より抽出される特徴量とそこからさら
に計算される点群データを処理できるだけの計算リソースが必要であった.
他にTDOAなどの手法を用いる場合には電波の送信機と受信機などの機材が必要であった.
さらにVisual/LiDAR SLAMを利用する場合は利用可能な環境に制約が発生する.\ref{slam_problem}

一方で提案手法ではトイドローンにも積載されている単眼カメラとQRコードの検出とQRコードの姿勢推定さえ
可能な計算リソースがあればQRコードを印刷するだけで最低限の明るさが確保出来れば環境による制約もなく利用可能である.

\subsection{各アプローチとの比較}
以下に既存手法との比較を表にしたものを示す.
比較対象はLiDAR SLAM, Visual SLAM, TDOAを対象とする.

ここに表をいれる
