\subsection{経路情報の管理}
\label{route_problem}

既存の方法による経路情報の管理はいくつか方法が考えられる.

はじめに考えられるのは機体それぞれに飛行プログラムの一部として経路情報を書き込む方法である.
外部との通信を必要とせず自律飛行が可能である事に加えて,倉庫の巡回タスクにおいては基本的に経路情報が変化しない為適していると考えられる.
しかし飛行前に経路情報のチェックやアップデートの有無の確認,経路情報のバージョン管理はやはり必要であると考えられる.
また,機器の故障やメンテナンスによる機体の入れ替えの度に飛行経路に応じた書き換えが必要となる.
同じ経路のプログラムを書き込んだ予備機体を用意することも考えられるが飛行させたい経路が増えるにつれてその予備台数も増加する事になってしまう.

次に考えられるのがグラウンドコントロールシステムによる中央集権的な管理である.
基本の飛行システムさえ書き込んでおけば経路情報と誘導はグラウンドコントールシステム側で一元管理が可能になり
飛行前のアップデートの確認の手間やバージョン管理は不要になると考えられる.
機体側は外部通信機構が必須になる上グラウンドコントロールシステム側での複数機体の管理,
グラウンドコントロールシステムと各機体とのコミュニケーションが必要となる上に
機体数が増えた場合にそのネットワークの帯域も問題となる可能性があるというデメリットがある.