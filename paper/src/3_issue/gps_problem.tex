\subsection{GPSを用いた自己位置推定}
現在世界座標系における自己位置推定を行う上で頻繁に用いられるが,その利用可能環境には制限がある.
それは屋内や遮蔽物の多い環境で衛星電波を受信出来ず利用できないという事である.
加えて屋外の見通しの良い環境であっても誤差が平均して1~2m程あり\footnote{同研究室の水野史暁氏と竹瀬浩行氏と他数名によって慶應義塾大学湘南藤沢キャンパスWEST地区にて計測}細かい制御には向かない.

一方で現在GPSの精度向上や補助をする為のシステムが複数開発されている.
その中の一つがA-GPS\cite{agps}と呼ばれ主に携帯電話やスマートフォンに搭載されているシステムである.
A-GPSはGPSの信号強度が弱い場合や,建造物などの影響によるマルチパスが発生している際に利用可能なネットワークを利用して測位者の概ねの位置を取得する方式である.これによりGPS測位時間の短縮や測位可能なエリアの拡大を実現している.

他にJAXAが開発したIMES(Indoor MEssaging System)\cite{imes}という屋内にGPSと同じ電波を利用した送信機を設置し細かい位置測位を行うという物もある.
しかしこれらの方法は専用の送信機や通信機構を用意する必要がある.