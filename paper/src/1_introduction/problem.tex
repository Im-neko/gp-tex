\section{本研究の問題と仮説}
現在自己位置推定にはGPS(Global Positioning System)を用いる手法が非常に有力であり頻繁に用いられる.
しかし,屋内等の理由により直上方向に遮蔽物がある場合にはその精度が著しく
低下することや利用できないという状況が発生する.

本研究では倉庫での物品管理に利用するドローンの為のナビゲーションシステムの開発を目標としている.
倉庫で物品管理にドローンを用いる上で求められるものとして第一に業務利用する以上そのメンテナンス性が挙げられる.

ハードウェア面でのメンテナンスはもとよりソフトウェア面でのメンテナンス性は非常に重要である.
倉庫内を複数機で巡回する形式のドローンの運用を考えるとその経路情報の管理はいくつか方法が考えられる.
はじめに考えられるのは機体それぞれに飛行プログラムの一部として経路情報を書き込む方法である.
外部との通信を必要とせず自律飛行性能は高くなるというメリットがあるが,
常に最新の経路情報で飛行させようとすれば経路情報のバージョン管理や飛行前のアップデート確認の手間が必要となるというデメリットがある.
次に考えられるのがグラウンドコントロールシステムによる中央集権的な管理である.
基本の飛行システムさえ書き込んでおけば経路情報と誘導はグラウンドコントールシステム側で一元管理が可能になり
飛行前のアップデートの確認の手間は不要になるというメリットがあるが.
機体側は外部通信機構が必須になる上グラウンドコントロールシステム側での複数機体の管理,
グラウンドコントロールシステムと各機体とのコミュニケーションが必要となる上に
機体数が増えた場合にそのネットワークの帯域も問題となる可能性があるというデメリットがある.

加えてこれら方式では機体数が増えるに連れてそれぞれのデメリットは大きな物になる上,
管理対象の規模が大きくなる場合その経路情報も同時に変更になると考えられ,
規模変更に合わせたシステム自体の修正が必要になる.

そこで今回は機体数の増加による管理コストの増加を抑えられるように経路情報に関してステートレスな
ナビゲーションシステムをQRコードを用いて開発する.