\section{本研究の問題と仮説}
自律航行するロボットの制御において最も重要な要素のうちの一つが自己位置推定である.
自己位置推定とは,任意の系においてロボット自身がどこにいるのかを推定する事である.

現在自己位置推定にはGPS(Global Positioning System)を用いる手法が非常に有力であり頻繁に用いられる.
しかし,屋内等の理由により直上方向に遮蔽物がある場合にはその精度が著しく低下することや利用できないという状況が発生する.
その為屋内では後述するVisual SLAMやLiDAR SLAM,TDOA等様々な自己位置推定手法が考案され利用されている.
これらの手法を利用したナビゲーションシステムで倉庫の巡回タスクを想定した実運用を考えると,
いくつかの冗長な点と問題点が浮上する.

\subsection{倉庫の巡回タスクにおける特徴}
まず倉庫の巡回というタスクを考えた際に挙げられる特徴を以下にまとめる.
\begin{itemize}
    \item 移動経路がほぼ固定化する
    \item 障害物の位置が既知である
    \item 機器の故障やメンテナンス等により機体の入れ替えが頻繁に発生する
    \item 収容率を上げる為にも通路は狭ければ狭い程良いとされる
\end{itemize}

\subsection{倉庫の巡回タスクにおける課題}
これらの特徴を踏まえた上で既存のナビゲーションシステムでは冗長であると考えられる部分や課題を以下にまとめる.
詳細については\ref{issue}章にて説明する.
\begin{itemize}
    \item 経路情報の管理 \ref{route_problem}
    \item 倉庫内での自己位置推定 \ref{map_problem}
    \item 専用機材の用意 \ref{equipment_problem}
\end{itemize}

\subsection{本研究の仮説}
そこで本研究ではQRコードを利用する事で倉庫の巡回タスクの特徴を踏まえて既存の自己位置推定方法を利用せず,
また経路情報のソフトウェア側での管理も行わず,
比較的容易に用意できる機材を用いて巡回タスクを達成する事ができるのではないかと考えた.