\section{本研究の問題と仮説}
現在自己位置推定にはGPS(Global Positioning System)を用いる手法非常に有力でありよく用いられる.
しかし,屋内等の理由により直上方向に遮蔽物がある場合にはその精度が著しく低下することや利用できないという状況が発生する.

本研究では倉庫での物品管理に利用するドローンの為のナビゲーションシステムを想定したシステムの開発を目標としている.
ドローンを物品管理に利用するメリットとしては,現状物品管理は人力や物品棚自体にレールを設置しそのレール上を管理機械が移動しながら管理する方法が取られることが多い.
一方でドローンを用いた物品管理を行えばレールの敷設コスト等を支払わずに自動管理が可能となる.

ここでドローンを用いた物品管理の際のドローンの運用方法を考えると効率化の為には並列して複数機体をそれぞれ異なる経路で定期的に飛行させることが想定される.
この時各機体毎の経路情報管理コストは規模と機体数に応じて大きな物になる.また一般的に規模に合わせて機体数も増えると考えられる.
そこで今回は機体数の増加による管理コストの増加を抑えられるように経路情報に関してステートレスなナビゲーションシステムをQRコードを用いて開発する.

ドローンを自律飛行させる際に重要となるのが自己位置推定と経路選択である.
ドローンの飛行システムに経路情報を書き込む場合機体をメンテナンスする等の理由で別機体と入れ替える場合もその度に経路情報の更新が必要になる.
加えて飛行させたい経路が変更になった際にも全てのドローンに対して経路情報を更新する必要がある