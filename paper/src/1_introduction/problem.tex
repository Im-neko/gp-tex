\section{本研究の問題と仮説}
現在自己位置推定にはGPS(Global Positioning System)を用いる手法が非常に有力であり頻繁に用いられる.
しかし,屋内等の理由により直上方向に遮蔽物がある場合にはその精度が著しく
低下することや利用できないという状況が発生する.
その為屋内では後述するVisual SLAMやLiDAR SLAM,他にTDOAを用いた自己位置推定手法が利用されている.

実運用を想定したロボットにおいて重要な要素としてメンテナンス性が挙げられる.

その一つ目が経路情報の管理である.
倉庫内を複数機で巡回する形式のドローンの運用を考えるとその経路情報の管理はいくつか方法が考えられる.
はじめに考えられるのは機体それぞれに飛行プログラムの一部として経路情報を書き込む方法である.
外部との通信を必要とせず自律飛行性能は高くなるというメリットがあるが,
常に最新の経路情報で飛行させようとすれば経路情報のバージョン管理や飛行前のアップデート確認の手間が必要となるというデメリットがある.
次に考えられるのがグラウンドコントロールシステムによる中央集権的な管理である.
基本の飛行システムさえ書き込んでおけば経路情報と誘導はグラウンドコントールシステム側で一元管理が可能になり
飛行前のアップデートの確認の手間は不要になるというメリットがあるが.
機体側は外部通信機構が必須になる上グラウンドコントロールシステム側での複数機体の管理,
グラウンドコントロールシステムと各機体とのコミュニケーションが必要となる上に
機体数が増えた場合にそのネットワークの帯域も問題となる可能性があるというデメリットがある.

二つ目が機体自体の管理である.
上述したようなシステムでは機体の自己位置推定を行う上で専用の機材や画像から特徴点抽出及び環境地図の三次元再構成を行えるような計算リソースが必要となる.

また,現在これらの要件を満たすドローンを専用基板などを発注せずにOSSや各社より販売される開発用フライトコントローラ\footnote{ドローンが飛行する上で姿勢維持や飛行命令を行う装置}を利用して自作しようとすれば,充分な積載量を確保するためには一辺40cm程の機体サイズとなる.
倉庫のように人と同じ空間を飛行するドローンとしては少々恐怖感を与えるサイズとなる上事故時の被害もトイドローンのような10cm代の小型のものより大きくなる.

そこで今回は管理対象の規模変更によるシステムの修正を減らす為に経路情報に関してステートレスな
ナビゲーションシステムを開発出来ればこれらのデメリットは解消すると考えた.
本研究ではQRコードを用いて経路情報に関してステートレスなナビゲーションシステムの実現をする.
