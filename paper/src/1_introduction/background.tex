\section{本研究の背景}
\label{background}
\subsection{自己位置推定手法の発展}
モバイルデバイスの高度化やセンサの小型化,計算能力の向上により高度な処理が可能となった.
これにより自己位置推定の様々な手法が提案され,その精度も向上を続けている.
しかしながら実運用を考慮する上で運用コストや機体性能,環境による制約が多く要件に応じた適切な自己位置推定手法を取る必要がある.

\subsection{倉庫内での物品管理方法}
現在倉庫内の物品管理\footnote{本論文では定期的に物品の位置の記録を行うことを指す}に用いられる方法はいくつかあり,自動型の物であれば物品棚自体に設置された
レール上を物品管理,物品の配置/取得する専用機が移動するという形が取られている.この形を基本形として取得後はベルトコンベアーで移動,梱包等まで自動で行う倉庫もある.
一方で人間が専用機を用いて物品管理や物品の配置/取得をする形式の倉庫も多く存在している.
現在人間が物品管理を行っている倉庫で新たにレールを用いた形で自動化を行おうとするとレールの敷設が必要となる.
ここでレールの敷設が不要な自動物品管理の方法としてロボットを巡回させる形での物品管理の自動化が考えられる.
また,この際用いられるロボットとしては自由度の高いドローンを用いるのが効率的である.
そこで本研究では将来的に物品管理に利用されることを想定したドローンのナビゲーションシステム\footnote{本論文ではロボットの自己位置推定と経路選択を行うシステムを指す}の開発を目指す.

% 諸々まとめて管理コストとして表現したら楽だけど,
% コストの定義をしなければならないので余裕があればやる
