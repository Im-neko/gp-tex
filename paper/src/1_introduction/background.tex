\section{本研究の背景}
\label{background}
\subsection{自己位置推定手法の発展}
モバイルデバイスの高度化やセンサの小型化,計算能力の向上により高度な処理が可能となった.
これにより自己位置推定の様々な手法が提案され,その精度も向上を続けている.
しかしながら実運用を考慮すると運用に必要な金銭的費用や機体性能,環境による制約が多く要件に応じた適切な自己位置推定手法を取る必要がある.

\subsection{倉庫内での物品管理方法}
倉庫内の物品管理\footnote{本論文では定期的に物品の位置の記録を行うことを指す}の方法は大別すると以下の二通りが挙げられる.
一つ目が物品棚自体に機械を搭載したり,物品棚自体をロボットで移動させながら物品管理を行う方法.
二つ目は倉庫内は何も手を加えずに巡回ロボットを用いて物品管理を行う方法である.
自動型の物であれば物品棚自体に自動化用の機械を取り付ける場合や物品棚自体をロボットで移動させながら物品管理及び取得・移動を行う物が存在している.
一方で人間が読み取り機を用いて物品管理や物品の配置/取得をする形式の倉庫も多く存在している.
現在人間が物品管理を行っている倉庫で新たに形で自動化を行おうとすると倉庫自体の大きな改修が必要となる.
ここで大規模な改修が不要な物品管理の自動化の方法として自律航行するロボットを巡回させる形での物品管理の自動化が考えられる.
また,この際用いられるロボットとしては高さのある棚に対しても対応が容易な自由度の高いドローンを用いることが想定される.
そこで本研究では将来的に物品管理に利用されることを想定したドローンのナビゲーションシステム\footnote{本論文ではロボットの自己位置推定と経路選択を行うシステムを指す}の開発を目指す.

