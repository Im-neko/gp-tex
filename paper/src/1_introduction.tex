\chapter{序論}
\label{introduction}

本章では本研究の背景,課題及び手法を提示し,本研究の概要を示す.

\section{はじめに}
ロボットにおけるナビゲーションシステム\footnote{本論文では自己位置推定と経路選択を行い,ロボット誘導するシステムのこととする}の開発は自動操縦を行う上で非常に重要とな要素である.またその中でも自己位置推定の手法や経路選択のアルゴリズムは多岐にわたって発明されてきた.
2010年頃にParrot社\footnote{フランスに拠点を置くドローンメーカ}からAR Drone\footnote{個人向けの空撮用ドローン}が発売されるのを皮切りにドローンはその名前と共に国内外問わずその存在感を増している.
普及と共に個人での空撮目的での所有や産業用に測量や物品輸送,その他には人が立ち入るのが困難な場所での調査に向けて日々開発が行われている.

本研究では倉庫での物品管理に利用するドローンの為のナビゲーションシステムを想定したシステムの開発を目標としている.
ドローンを物品管理に利用するメリットとしては,現状物品管理は人力や物品棚自体にレールを設置しそのレール上を管理機械が移動しながら管理する方法が取られることが多い.
一方でドローンを用いた物品管理を行えばレールの敷設コスト等を支払わずに自動管理が可能となる.

ここでドローンを用いた物品管理の際のドローンの運用方法を考えると効率化の為には並列して複数機体をそれぞれ異なる経路で定期的に飛行させることが想定される.
この時各機体毎の経路情報管理コストは規模と機体数に応じて大きな物になる.また一般的に規模に合わせて機体数も増えると考えられる.
そこで今回は機体数の増加による管理コストの増加を抑えられるように経路情報に関してステートレスなナビゲーションシステムをQRコードを用いて開発する.

\section{本論文の構成}

本論文における以降の構成は次の通りである.

\ref{introduction}章では,導入を述べる.

\ref{background}章では,背景を述べる.

\ref{issue}章では,本研究における問題の定義と,解決するための要件の整理を行う.

\ref{proposed}章では,本研究の提案手法を述べる.

\ref{implementation}章では,~\ref{proposed}章で述べたシステムの実装について述べる.

\ref{evaluation}章では,\ref{issue}章で求められた課題に対しての評価を行い,考察する.

\ref{conclusion}章では,本研究のまとめと今後の課題についてまとめる.


%%% Local Variables:
%%% mode: japanese-latex
%%% TeX-master: "../thesis"
%%% End:
