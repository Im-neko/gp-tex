\chapter{序論}
\label{introduction}

本章では本研究の背景,課題及び手法を提示し,本研究の概要を示す.

\section{はじめに}
\label{introduction:background}
ロボットにおけるナビゲーションシステムの開発は自動操縦を行う上で非常に重要とな要素である.またその中でも自己位置推定の手法や経路選択のアルゴリズムは多岐にわたって発明されてきた.
2010年頃にParrot社\footnote{フランスに拠点を置くドローンメーカ}からAR Drone\footnote{個人向けの空撮用ドローン}が発売されるのを皮切りにドローンはその名前と共に国内外問わずその存在感を増している.
今まで多くのロボット向けのナビゲーションシステムは陸上の移動を前提に作られているものが多く,そのままドローンに導入しようとするとその特性故に導入コストが陸上移動前提のロボットよりも高くなる.
最も大きな違いとして積載物の重さによって航続可能時間が大きく変わる為非常に慎重にならなければならない事である.2019年11月にDJI社\footnote{中国に本社を置くドローンメーカ}より発売されたMavicMiniを例にしてみると日本国内発売機(本体重量199g)での本来の最大飛行時間が18分なのに対して,23.5gのプロペラガードを装着すると12分まで最大飛行可能時間が短くなる\footnote{最大飛行時間と重さはDJI社の技術仕様ページに記載されている.}とされている.

 
% mavic mini仕様: https://www.dji.com/jp/mavic-mini
% プロペラガード仕様: https://store.dji.com/jp/product/mavic-mini-360-propeller-guard


% これは単純にその場で停止する状況を考えれば明白ではあるが4輪のロボットを例に取ると4輪のロボットはモータが動かないようにロックするだけで良いのに対し,ドローンは常にモータを回転させ姿勢維持し続けなければならない.また地上では設置面との摩擦があるのに対し空中では空気抵抗しかなく,風などの外力を考慮した場合の姿勢維持に必要なエネルギーは更に大きくなる.

個人での空撮目的での所有や産業用に測量や物品輸送,その他人が立ち入るのが困難な場所での調査に向けて日々開発が行われている.
本研究はGPSが利用できないような環境での自律飛行制御を想定している。
ここでいうGPSが利用できない環境というのは地理的要因と機体的要因の2つがある。
地理的要因としては屋内や橋の下など遮蔽物が頭上にある環境、機体的要因としてはそもそも機体にGPSモジュールが無い、積載ペイロードが無い。
などが挙げられる。

こうした地理環境においてもカメラは問題なく使え、機体にもカメラが付いていることは多い。
これらのことを踏まえ、今回カメラでの自己位置推定手法を考えるに至った。
その上でカメラで認識が簡単に行えるものとしてQRコードを採用した。
QRコードは汚れや傷みにも非常に強くまた、ある程度のサイズのデータの保存も可能である。
更にQRコードは非対称な図面で正方形のため、マーカーとしての向きや歪みの検出が非常に容易に行える。
これらの特徴を利用し、QRコードをマーカー件情報伝達媒体として採用した自律飛行手法を検討する。


\section{本論文の構成}

本論文における以降の構成は次の通りである.

\ref{introduction}章では,導入を述べる.
\ref{background}章では,背景を述べる.
\ref{issue}章では,本研究における問題の定義と,解決するための要件の整理を行う.
\ref{proposed}章では,本研究の提案手法を述べる.
\ref{implementation}章では,~\ref{proposed}章で述べたシステムの実装について述べる.
\ref{evaluation}章では,\ref{issue}章で求められた課題に対しての評価を行い,考察する.
\ref{conclusion}章では,本研究のまとめと今後の課題についてまとめる.


%%% Local Variables:
%%% mode: japanese-latex
%%% TeX-master: "../thesis"
%%% End:
