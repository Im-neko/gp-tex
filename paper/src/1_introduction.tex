\chapter{序論}
\label{introduction}

本章では本研究の背景,課題及び手法を提示し,本研究の概要を示す.

\section{はじめに}
ロボットにおけるナビゲーションシステム\footnote{本論文では自己位置推定と経路選択を行い,ロボット誘導するシステムのこととする}の開発は自動操縦を行う上で非常に重要とな要素である.またその中でも自己位置推定の手法や経路選択のアルゴリズムは多岐にわたって発明されてきた.
2010年頃にParrot社\footnote{フランスに拠点を置くドローンメーカ}からAR Drone\footnote{個人向けの空撮用ドローン}が発売されるのを皮切りにドローンはその名前と共に国内外問わずその存在感を増している.
普及と共に個人での空撮目的での所有や産業用に測量や物品輸送,その他には人が立ち入るのが困難な場所での調査に向けて日々開発が行われている.

% A案
% 本研究ではGPSが利用できない倉庫のような屋内環境下でのドローンの自己位置推定と経路選択の方法について考察,実装を行う.
% また,倉庫内には棚が複数設置されているものとし棚間の通路を飛行できるようなナビゲーションシステムの開発を目指す.
% 屋内で且つ棚が複数設置されているような環境ではGPSが使えない事に加えて機体サイズが小さいことも求められる.
% 一般にMAV(Micro Airial Vehicle)\footnote{狭義ではDARPA(アメリカ国防高等研究計画局)が150mm以下のUAVを指すと定義している}と呼ばれるような小型のドローンであってもカメラは積載されていることが多い.
% この点に着目し,本研究では小型のドローンでのカメラを利用したナビゲーションシステムを開発する.
% その上でカメラで認識が簡単に行えるものとしてQRコードを採用した.
% QRコードは汚れや傷みにも非常に強くまた,ある程度のサイズのデータの保存も可能である.
% 更にQRコードは非対称な図面で正方形のため,マーカーとしての向きや歪みの検出が非常に容易に行える.
% これらの特徴を利用し,QRコードをマーカー件情報伝達媒体として採用した自律飛行手法を検討する.

本研究では倉庫での物品管理に利用するドローンの為のナビゲーションシステムの開発を目標としている.
ドローンを物品管理に利用するメリットとしては,現状物品管理は人力や物品棚自体にレールを設置しそのレール上を管理機械が移動しながら管理する方法が取られることが多い.
一方でドローンを用いた物品管理を行えばレールの敷設コスト等を支払わずに自動管理が可能となる.
ここでドローンを用いた物品管理の方法を考えるとドローンを定期的に飛行させる必要があり,効率化の為に並列して複数機体をそれぞれ異なる経路で飛行させることが想定される.
この時各機体毎の経路情報管理コストは規模に応じて大きなものとなる.
そこで今回は経路情報に関してステートレスなナビゲーションシステムをQRコードを用いて開発する.




\section{本論文の構成}

本論文における以降の構成は次の通りである.

\ref{introduction}章では,導入を述べる.
\ref{background}章では,背景を述べる.
\ref{issue}章では,本研究における問題の定義と,解決するための要件の整理を行う.
\ref{proposed}章では,本研究の提案手法を述べる.
\ref{implementation}章では,~\ref{proposed}章で述べたシステムの実装について述べる.
\ref{evaluation}章では,\ref{issue}章で求められた課題に対しての評価を行い,考察する.
\ref{conclusion}章では,本研究のまとめと今後の課題についてまとめる.


%%% Local Variables:
%%% mode: japanese-latex
%%% TeX-master: "../thesis"
%%% End:
