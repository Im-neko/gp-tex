\chapter{序論}
\label{introduction}

本章では本研究の背景,課題及び手法を提示し,本研究の概要を示す.

\section{はじめに}
\label{introduction:background}
2010年頃にParrot社\footnote{フランスに拠点を置くドローンメーカ}からAR Drone\footnote{個人向けの空撮用ドローン}が発売されるのを皮切りにドローンはその名前と共に国内外問わずその存在感を増してきた.
個人での空撮目的での所有や産業用に測量や物品輸送,その他人が立ち入るのが困難な場所での調査に向けて日々開発が行われている.

本研究はGPSが利用できないような環境での自律飛行制御を想定している。
ここでいうGPSが利用できない環境というのは地理的要因と機体的要因の2つがある。
地理的要因としては屋内や橋の下など遮蔽物が頭上にある環境、機体的要因としてはそもそも機体にGPSモジュールが無い、積載ペイロードが無い。
などが挙げられる。

こうした地理環境においてもカメラは問題なく使え、機体にもカメラが付いていることは多い。
これらのことを踏まえ、今回カメラでの自己位置推定手法を考えるに至った。
その上でカメラで認識が簡単に行えるものとしてQRコードを採用した。
QRコードは汚れや傷みにも非常に強くまた、ある程度のサイズのデータの保存も可能である。
更にQRコードは非対称な図面で正方形のため、マーカーとしての向きや歪みの検出が非常に容易に行える。
これらの特徴を利用し、QRコードをマーカー件情報伝達媒体として採用した自律飛行手法を検討する。


\section{本論文の構成}

本論文における以降の構成は次の通りである.

\ref{introduction}章では,導入を述べる.
\ref{background}章では,背景を述べる.
\ref{issue}章では,本研究における問題の定義と,解決するための要件の整理を行う.
\ref{proposed}章では,本研究の提案手法を述べる.
\ref{implementation}章では,~\ref{proposed}章で述べたシステムの実装について述べる.
\ref{evaluation}章では,\ref{issue}章で求められた課題に対しての評価を行い,考察する.
\ref{conclusion}章では,本研究のまとめと今後の課題についてまとめる.


%%% Local Variables:
%%% mode: japanese-latex
%%% TeX-master: "../thesis"
%%% End:
