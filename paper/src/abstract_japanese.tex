卒業論文要旨 - 2019年度 (令和元年度)
\begin{center}
\begin{large}
\begin{tabular}{|M{0.97\linewidth}|}
    \hline
      \title \\
    \hline
\end{tabular}
\end{large}
\end{center}
~ \\
人間が物品管理を行っている倉庫で物品管理の自動化を行おうとする場合二通りの方法が考えられる.
一つ目が物品棚自体に機械を搭載したり,物品棚自体をロボットで移動させながら物品管理を行う方法.
二つ目は倉庫内は何も手を加えずに巡回ロボットを用いて物品管理を行う方法である.
本研究では2つ目の方法で物品管理に利用されることを想定したロボットのナビゲーションシステムの開発を目指す.
ロボットには移動の自由度が高いドローン(小型無人航空機)を採用している.
ロボットを自律的に倉庫内を移動させる場合ロボットの経路情報の管理は非常に重要であり,
現状ロボットのナビゲーションシステムでは経路情報はソフトウェアによる生成及び管理が主である.
しかしながら,倉庫での巡回という目的においては現状のナビゲーションシステムは経路情報の扱いを含めて少々冗長な部分が存在している.
そこで今回は特に経路情報の扱いに注目し,経路情報に関してステートレスなナビゲーションシステムを考案し利用可能かを検証した.
本研究ではQRコードを用いて経路情報に関してステートレスなナビゲーションシステムの実現をする.
~ \\
キーワード:\\
\underline{1. ドローン},
\underline{2. ナビゲーションシステム},
\underline{3. 倉庫},
\underline{4. QRコード},
\begin{flushright}
\dept \\
\author
\end{flushright}
