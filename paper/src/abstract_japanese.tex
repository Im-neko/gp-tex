卒業論文要旨 - 2019年度 (令和元年度)
\begin{center}
\begin{large}
\begin{tabular}{|M{0.97\linewidth}|}
    \hline
      \title \\
    \hline
\end{tabular}
\end{large}
\end{center}
~ \\
人間が物品管理を行っている倉庫で物品管理の自動化を行おうとする場合二通りの方法が考えられる.
一つ目が倉庫内にレールを設置しレール上を専用機が移動しながら物品管理を行う方法.
二つ目は倉庫内は何も手を加えずに巡回ロボットを用いて物品管理を行う方法である.
本研究では2つ目の方法で物品管理に利用されることを想定したロボットのナビゲーションシステムの開発を目指す.
ロボットには移動の自由度が高いドローン(小型無人航空機)を採用している.
その上でロボットの経路情報の管理は非常に重要であり,現状ロボットのナビゲーションシステムでは経路情報はソフトウェアによる管理が主である.
一方で管理対象の規模が大きくなる場合その経路情報も同時に変更になると考えられ,
規模変更に合わせてシステム自体の修正が必要になる.
そこで今回は管理対象の規模変更によるシステムの修正を減らす為に経路情報に関してステートレスな
ナビゲーションシステムを開発出来ればこれらのデメリットは解消すると考えた.
本研究ではQRコードを用いて経路情報に関してステートレスなナビゲーションシステムの実現をする.
システム構築を行う上ではROSを採用し使用するロボットが変化した場合にも対応が容易となるような設計を行う.
~ \\
キーワード:\\
\underline{1. ドローン},
\underline{2. ナビゲーションシステム},
\underline{3. 倉庫},
\underline{4. QRコード},
\underline{5. ROS},
\begin{flushright}
\dept \\
\author
\end{flushright}
