卒業論文要旨 - 2019年度 (令和元年度)
\begin{center}
\begin{large}
\begin{tabular}{|M{0.97\linewidth}|}
    \hline
      \title \\
    \hline
\end{tabular}
\end{large}
\end{center}
~ \\
現在人間が物品管理を行っている倉庫で新たにレールを用いた形でレールを用いた物品管理の自動化を行おうとするとレールの敷設が必要となる.
レールの敷設が不要な物品管理の自動化としてドローンを用いた物品管理の自動化が考えられる.
そこで本研究では将来的に物品管理に利用されることを想定したドローンのナビゲーションシステムの開発を目指す.
その上でドローンの経路情報の管理は非常に重要であり,現状ロボットのナビゲーションシステムでは経路情報はソフトウェアによる管理が主である.
一方で管理対象の規模が大きくなる場合その経路情報も同時に変更になると考えられ,
規模変更に合わせてシステム自体の修正が必要になる.

倉庫での経路情報は動的には変わらないものの,棚の追加や配置変更などである程度変更されることが想定される.
そこで今回は管理対象の規模変更によるシステムの修正を減らす為に経路情報に関してステートレスな
ナビゲーションシステムを開発出来ればこれらのデメリットは解消すると考えた.
本研究ではQRコードを用いて経路情報に関してステートレスなナビゲーションシステムの実現をする.
~ \\
キーワード:\\
\underline{1. ドローン},
\underline{2. ナビゲーションシステム},
\underline{3. 倉庫},
\underline{4. QRコード},
\underline{5. ROS},
\underline{6. SLAM},
\begin{flushright}
\dept \\
\author
\end{flushright}
