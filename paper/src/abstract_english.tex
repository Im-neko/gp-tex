Abstract of Bachelor's Thesis - Academic Year 2020
\begin{center}
\begin{large}
\begin{tabular}{|p{0.97\linewidth}|}
    \hline
      \etitle \\
    \hline
\end{tabular}
\end{large}
\end{center}

~ \\
There is 2 way which realizes automate item management system in the warehouse.  
The first way is putting some rail and using dedicated robots on the rail.
The second way is using patrol robots without any warehouse reform.  
This research focuses on the robot which is used in a second way.
We use a drone as a robot because the drone has a high degree of freedom.  
Moreover, It is important that is route information and is managed by software mainly.
On the other hand, we need to update or install navigation software when the route information or robot has changed.
Furthermore, the software is going to large when routes increased.
Thus, we propose a navigation system which is stateless of route information.
In this system, we use QRCode to make it stateless of route information.
We use ROS to build this system because making it easy to change the system when a robot is going to change.
~ \\
Keywords : \\
\underline{1. Drone},
\underline{2. Navigation System},
\underline{3. WareHouse},
\underline{4. QRCode},
\underline{5. ROS},
\begin{flushright}
\edept \\
\eauthor
\end{flushright}
