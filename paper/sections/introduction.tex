\section{はじめに}
本研究はGPSが利用できないような環境での自律飛行制御を想定している。
ここでいうGPSが利用できない環境というのは地理的要因と機体的要因の2つがある。
地理的要因としては屋内や橋の下など遮蔽物が頭上にある環境、機体的要因としてはそもそも機体にGPSモジュールが無い、積載ペイロードが無い。
などが挙げられる。

こうした地理環境においてもカメラは問題なく使え、機体にもカメラが付いていることは多い。
これらのことを踏まえ、今回カメラでの自己位置推定手法を考えるに至った。
その上でカメラで認識が簡単に行えるものとしてQRコードを採用した。
QRコードは汚れや傷みにも非常に強くまた、ある程度のサイズのデータの保存も可能である。
更にQRコードは非対称な図面で正方形のため、マーカーとしての向きや歪みの検出が非常に容易に行える。
これらの特徴を利用し、QRコードをマーカー件情報伝達媒体として採用した自律飛行手法を検討する。

\subsection{研究目的}
カメラが搭載されている機体で、移動制御が行えるという事を条件にした自律制御を目的とする。
また、本手法はカメラ映像と移動制御が自由に行える機体であればどのような機体であっても利用できることを目指す。

また、GPSが利用できない倉庫等の屋内での自己位置推定を行い、安定した自律飛行をできるようにする事を目的としている。
