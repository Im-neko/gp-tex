\section{はじめに}
ドローンをRGBカメラで制御しようという考えに至った理由として,映像処理技術の向上,ドローン製作部品が現時点では比較的高価格であること,があげられる.高度維持やGPSによる位置補正など一般的な機能を備えた安定飛行できるドローンを一機製作しようとすれば,センサーの類だけで数万円はかかってしまう.
そこで、既に小型化、低価格化が進んでいるRGBカメラを用いた画像処理による飛行経路探索を行う.
加えて、LIDARなどのセンサーで得られる情報で障害物等自体の意味解釈まで行う事は困難である。そこで、RGBカメラを用いて周囲の映像を取得し周辺にどのような物体が存在しているのかを検出しながらの飛行が求められる。

また、本研究では飛行モデルの提案を軸としており物体検出モデルの提案ではないということをここで明記させていただきたい。


\subsection{背景}
2010年頃からParrot社による家庭用ドローンの発売などもあり,これ以降ドローンの認知と研究開発は加速している.
付随して,自律飛行に関しても研究が盛んに行われているが,現在行われている自律飛行ではLIDARなど他のセンサーと比較して高価なものを使用している例が多い.

また広く流通しているものは大きいものばかりである.これらの問題はドローンでの活用がさらに進めばセンサー自体の開発も進み解決するだろう.しかし,LIDAR等のセンサー以外を主とした飛行経路探索手法を模索することは今後の自律飛行の精度向上に貢献すると考え,本研究を実施するに至った.
また、本研究の課題の1つとしてLIDAR等の他のセンサー類が高価であったりまだ小型化が進んでいない事を挙げているが、センサー類の低価格化、小型化は日々進んでおり数年で課題とはならなくなると思われるが、その上でRGBカメラを他のセンサーと併用する上で周囲の物体検出は必要な技術であると考えらえる。


\subsection{研究目的}
本研究の目的はドローンの制御精度向上であり,本手法を単体で飛行経路選択をする以外にも,他のLIDARなどのセンサーも合わせて使用するような制御系における,制御の一助となるような使用を想定している.
また,RGBカメラの特性上雨天時や夜間などは著しく精度が低下する事が予測される為,基本的には屋内や,明るさがある程度確保できる条件下での使用を想定している.
その中で,本手法で用いる物体検知アルゴリズムではRGBカメラによる映像を使用する為,LIDAR等の他センサを利用するよりも安価に視野画内にある物体の意味を理解することができる.その点でLIDARなどの手法に比べて物体の特性なども考慮した判断をすることができる点が他の手法との異なる点である.
最終的には目の前にある物体が設置物なのか,移動物体なのか等を判断した上での飛行経路の選択をする事で屋内での飛行などでもより急な制御をする事なく事前に停止したり,回避行動をとれるようになると考えられる.
現在路面を走る車両には信号機がある為スムーズに飛行が行えるが,飛行物体に対する信号機は無い.しかし,今後様々な用途で飛び交うドローンが他のドローンと遭遇する機会が発生することは容易に想像でき,遭遇した際の事故を避ける技術が求められる.
そこで,本研究ではドローンに積載されたRGBカメラの映像からの情報を元に機体の飛行経路を選択するアルゴリズムを提案する.
