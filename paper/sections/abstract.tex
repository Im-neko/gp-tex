\begin{abstract}
2018年現在自律飛行で多く使用されるLIDARなどのセンサーは他の超音波やRGBカメラに比べて高価であり,またサイズも広く流通しているものは大きい物が多い.
ドローンなどの飛行物体では自動運転車などに比べて制動距離が大きく,空間に存在する他の移動物体を認識し,前もって飛行経路を変更,ないしは減速などを行う必要がある.
しかし,現在広く用いられているLIDARなどのセンサーでは物体の有無とその距離しか認識出来ず,その物体の意味解釈までは難しい.
そこで,本研究では比較的安価で手に入り小型の物も多く流通しているRGBカメラを利用して,周囲の物体の意味解釈を行う自律飛行制御を目指す.

また,今回は研究を進める上での状況設定として考慮事項が少なく,再現性のある環境を作れるようにする為にドローンレースを前提としている.
流れとしては実際にドローンレースで用いられるようなゲートを配置しそれらの理想形となる通過軌道を正としてシミュレーションを中心に学習を行う.
また,この時複数のコースを同時に学習させる事で汎用性を持たせられるという先行研究もあり本研究でも同様に進めるものとする.

\end{abstract}
