\begin{abstract}
2018年現時点で安価で広く流通している上に小型のものも多く存在するRGBカメラを用いた自律飛行経路探索手法を提案する.
現在自律飛行で多く使用されるLIDARなどのセンサーは他の超音波やRGBカメラに比べて高価であり,またサイズも広く流通しているものは大きい物が多い.
ドローンなどの飛行物体では自動運転車などに比べて制動距離が大きく,空間に存在する他の移動物体を認識し,前もって飛行経路を変更,ないしは減速などを行う必要がある.
しかし、現在広く用いられているLIDARなどのセンサーでは物体の有無とその距離しか認識出来ず、その物体の意味解釈までは難しい。
そこで,本研究では比較的安価で手に入り小型の物も多く流通しているRGBカメラを利用して,周囲の物体の意味解釈を行う自律飛行制御を目指す.
本研究では屋内で複数機体が飛び交う環境での自律飛行を目標としている.
将来的には他の自律飛行手法の補助となるような技術として応用されることを想定している.

\end{abstract}
