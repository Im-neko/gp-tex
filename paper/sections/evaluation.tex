\section {評価}
評価に関しては段階をいくつかに分け、段階に応じた評価を行う。
まず第一段階は各地点同士の距離をどこまで離しても飛行可能かを評価する。
QRコード同士の距離が離れれば離れるほどマーカーの未認識区間が広くなり、自己位置推定が難しくなる。
そのため、QRコードでの経路制御の限界距離をここで記録し、距離とコース通りの飛行の成功率を計測する。
ここでの飛行の成功、不成功は以下のように定義する。

「任意のQRコードAの正面からスタートし、次の地点Bへ到達し、QRコードを5秒以上フレームインし続ける事が出来るか。」

という条件を以って成功、不成功を定義する。

上記の評価に加えて、安定して地点間の移動が可能になった場合に次の段階に入る。
次の段階では既存の他のシステムとの比較を行う。
周回コースを用意し、同じコースを飛行する上でQRコードを用いた自己位置推定システムとVO(Visual Odometry)での自己位置推定と比較する。
精度の比較はコースの周回速度と成功率で比較する。
成功率は上記での定義を同様に用いる物とする。
