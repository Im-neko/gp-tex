
\documentclass[a4j,10pt]{jsarticle}
\usepackage{layout,url,resume}
\usepackage[dvipdfmx]{graphicx}
\pagestyle{empty}

\begin{document}
%\layout

\title{[TERM]RGBカメラによる自律飛行経路探索手法の提案}

% 和文著者名
\author{
    著者その1 {井手田 悠希} \thanks{慶應義塾大学 環境情報学部 武田圭史研究室}
}

% 和文概要
\begin{abstract}
現時点で安価で広く流通している上に小型のものも多く存在するRGBカメラによる自律飛行経路探索手法を提案する.また,本研究では障害物が多く存在する屋内でのランダムウォークを第一段階の目標とし.また,これらに加えて本研究では人物追従や他のドローンの追従,追従しているドローンによる被追従ドローンの制御を目標最終的な到達目標としている.
\end{abstract}

\maketitle
\thispagestyle{empty}

\section{はじめに}
ドローンを映像で制御していくことの今後の展望として,ドローンの第三者視点映像による高精度制御がある.これは物資輸送を行う際に,複数機体のメンテナンスや冗長性を考慮して保有機種を少なくする為に,大型の物資は複数の中型機で輸送し,それ以外は1機で輸送したい.
これらを実現する上で複数ドローンが密集した状態での高精度制御が要求される.その前段階の技術として,ドローンをRGBカメラの映像によって制御することを実現したい.

\section{背景}
2010年頃からParrot社による家庭用ドローンの発売などもあり,これ以降ドローンの認知と研究開発は加速している.
付随して,自律飛行に関しても研究が盛んに行われているが,現在行われている自律飛行ではLIDARなど他のセンサーと比較して高価なものを使用している例が多い.また広く流通しているものは大きいものばかりである.これらの問題はドローンでの活用がさらに進めばセンサー自体の開発も進み解決するだろう.しかし,LIDAR等のセンサー以外を主とした飛行経路探索手法を模索することは今後の自律飛行の精度向上に貢献すると考え,本研究を実施するに至った.


\section{研究目的}
本研究の手法はドローンの制御精度向上であり,本手法を単体で飛行経路選択をする以外にも,他のLIDARなどのセンサーも合わせて使用するような制御系における,制御の一助となるような使用を想定している.
その中で,本手法で用いる物体検知アルゴリズムではRGBカメラによる映像を使用する為,LIDAR等の他センサを利用するよりも安価に視野画内にある物体の意味を理解することができる.その点でLIDARなどの手法に比べて物体の特性なども考慮した判断をすることができる点が他の手法との異なる点である.
最終的には目の前にある物体が設置物なのか,移動物体なのか等を判断した上での飛行経路の選択をする事で屋内での飛行などでもより急な制御をする事なく事前に停止したり,回避行動をとれるようになると考えられる.
現在路面を走る車両には信号機がある為スムーズに飛行が行えるが,飛行物体に対する信号機は無い.しかし,今後様々な用途で飛び交うドローンが他のドローンと遭遇する機会が発生することは容易に想像でき,遭遇した際の事故を避ける技術が求められる.
そこで,本研究ではドローンに積載されたRGBカメラの映像からの情報を元に機体の飛行経路を選択するアルゴリズムを提案する.

\section{関連研究}
関連研究として強く影響を受けているものが Nanomap\cite{Nanomap}である.
Nanomapにおいては2D-LIDARを用いた飛行経路探索手法が取られており,取得した点群データを過去数フレーム分を記憶しておき,過去の数フレーム分のデータと現在のフレームとの点群の変量によっておおよその障害物の位置を認識している.こちらの手法では,物体を正確に捉えることは放棄し,不確実な部分を物体の存在している可能性のある範囲として捉えている.視野にある物体のある可能性も含めて一番物体が少ない方向を飛行経路として選定し,飛行している.
他に,ドローンに限らずロボット工学全般で利用されてきたOctomap\cite{Octomap}がある.
こちらもNanomapと同様にLIDARを用いて周辺の環境情報を扱うものであるが,こちらはLIDARを用いて実際に周囲を点群からモデリングして周辺状況を把握する.
また,Octomapではモデリングする際に近い点群同士を同じ物体としてまとめ一つのブロックにするということを繰り返し,樹構造的に物体を生成する為,生成後のデータは繰り返す回数等のパラメータによっては小さく圧縮することもできる.
しかし,実際に点群からモデルを生成するとなると計算量が非常に多く高性能な小型コンピュータが現れている現在においても処理負荷が大きく,特にドローンにおいては積載量に制限がかかりやすい為扱いにくいものとなっている.


\section{提案手法}

\section{評価}

\section{考察}

\begin{thebibliography}{99}
%\bibitem{a}
\bibitem{Nanomap}
\texttt{Peter R. Florence1, John Carter1, Jake Ware1, Russ Tedrake1, NanoMap: Fast, Uncertainty-Aware Proximity Queries with Lazy Search over Local 3D Data}

\bibitem{Octomap}
\texttt{A. Hornung, K. M. Wurm, M. Bennewitz, C. Stachniss, and W. Burgard, “Octomap: An efficient probabilistic 3d mapping framework based on octrees,” Auton. Robots, vol. 34, no. 3, pp. 189–206, Apr. 2013. [Online]. Available: http://dx.doi.org/10.1007/ s10514- 012- 9321- 0}

\bibitem{Voxblox}
\texttt{H. Oleynikova, Z. Taylor, M. Fehr, J. Nieto, and R. Siegwart, “Voxblox: Building 3d signed distance fields for planning,” arXiv preprint arXiv:1611.03631, 2016.}

\bibitem{SfMDrone}
\texttt{此村 領, 堀 浩一, 実用性を備えた手のひらサイズ・完全オンボード処理 UAV のための 3 次元自己位置推定手法の提案と全自動飛行の実現, 東京大学工学系研究科航空宇宙工学専攻}

\end{thebibliography}

\end{document}
% end of file
